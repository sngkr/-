% !TeX root = ../main.tex

\chapter{持续交付系统总体架构(概要设计)与关键算法}
\label{chap:overview}

本章在前一章需求分析的基础上,从系统工程的视角出发,构建“面向多变更领域的持续交付系统”的总体架构方案与核心算法逻辑。针对多变更域环境下物理隔离与逻辑耦合并存的复杂现状,本章将详细阐述系统的逻辑分层架构、功能子系统详细设计以及支撑跨域协同的关键算法模型。作为连接需求与实现的桥垒,本章不仅解决了“如何构建系统”的工程架构问题,更通过科学的算法建模确保系统能够应对复杂业务场景下的交付稳定性与一致性挑战。

\section{总体架构设计}

系统的总体设计旨在为异构且高并发的变更任务提供一个统一的编排与调度底座。

\subsection{架构设计原则}

在多变更领域交叉影响的背景下,本系统遵循以下核心设计原则:
\begin{itemize}
    \item \textbf{高可用与自愈性 (High Availability)}:控制平面采用无状态副本部署,执行平面通过心跳租约机制确保任务在物理故障发生时的自动重试与迁移。
    \item \textbf{极致可扩展性 (Scalability)}:计算资源(Runner)支持基于任务积压水位的弹性伸缩,功能模块通过插件化框架实现热拔插。
    \item \textbf{严格的逻辑隔离 (Isolation)}:通过命名空间(Namespace)实现多租户间的资源与元数据物理隔离,各变更域(代码、配置、DB)执行环境完全独立。
    \item \textbf{端到端的零信任安全 (Zero Trust Security)}:所有 API 调用均需经过身份挑战,敏感凭证在容器启动时动态注入且阅后即焚。
\end{itemize}

\subsection{逻辑分层架构}

系统采用了典型的四层逻辑架构模型,自底向上分别为:
\begin{enumerate}
    \item \textbf{数据持久化层 (Data Layer)}:采用异构存储策略。MySQL 维护事务性元数据,Neo4j 存储跨域依赖图谱(MDDG),Prometheus 与 ELK 集成时序指标与海量日志。
    \item \textbf{分布式执行层 (Execution Layer)}:基于容器技术构建的弹性计算平面。Runner 节点负责具体检查插件与部署逻辑的落地,确保环境的一致性。
    \item \textbf{智能化编排层 (Orchestration Layer)}:系统的“大脑”。负责解析发布蓝图,动态生成有向无环图(DAG),并执行基于风险评分的门禁控制。
    \item \textbf{接入与控制层 (Gateway Layer)}:提供统一的 OpenAPI 与 UI 门户。负责异步信号的接入、协议转换、身份鉴权以及全局流量调度。
\end{enumerate}

\subsection{核心模块职能划分}

基于第三章的五大功能模块要求,系统内部职能被精确定位为五个协同工作的子系统:触发感知子系统(感知变化)、插件检查子系统(质量门禁)、编排中心(逻辑调度)、反馈监控子系统(实时闭环)及管理子系统(基础维护)。这种高内聚低耦合的设计确保了系统能够高效响应复杂变更信号。

\section{核心子系统详细架构设计}

\subsection{触发感知子系统}
感知子系统作为系统的“传感器”,其内部包含高性能 Webhook 适配器阵列与语义识别引擎。它通过订阅异构数据源(Git, Apollo, Nacos 等)的事件流,利用正则表达式与抽象语法树(AST)扫描技术,将非标注的推流信号转化为系统内部统一的“变更原子”描述符,并利用消息队列(MQ)缓解并发压力。

\subsection{插件化检查子系统}
为支撑灵活的检查能力,该子系统采用了 Sidecar 容器隔离模式。每个检查算子(如单元测试、Sonar 扫描、SQL 审计)均封装在独立的镜像中。子系统负责算子的版本管理、动态加载及运行时的 CPU/Memory 资源限制,确保第三方插件的故障不会传染给主执行链路。

\subsection{智能化编排中心设计}
编排中心负责将用户的交付意图转化为物理执行路径。它集成了 DAG 动态解析逻辑,能够根据 MDDG 计算出的隐含依赖自动补全任务节点。例如,当检测到代码变更依赖于特定的 DDL 脚本时,编排中心会自动在发布任务前置处插入数据库变更验证节点。

\subsection{全链路反馈与监控看板设计}
反馈中心实现了信息的“情景化推送”。它通过全量采集 Runner 节点的执行 Trace 数据,并实时拉取生产环境的业务监控指标(QPS/RT),构建基于 DORA 模型的效能仪表盘。其自诊断功能可为失败的流水线自动关联历史知识库,提供修复建议。

\subsection{统一身份认证与权限管理子系统}
子系统构建了基于 RBAC(角色权限控制)与 ABAC(属性权限控制)的级联鉴权框架。除对接企业 LDAP 统一身份外,还实现了基于时间窗口、风险等级的动态权限下发。例如,在“锁定期间”可以禁止所有非紧急变更的发布权限。

\section{数据持久化设计}

\subsection{关系型数据模型 (MySQL)}
MySQL 存储系统的“静态事实”。核心表包括:
\begin{itemize}
    \item \texttt{change\textunderscore records}:存储全局唯一的变更任务状态与风险评分。
    \item \texttt{pipeline\textunderscore snapshots}:保存 DAG 的序列化定义与配置快照。
    \item \texttt{atomic\textunderscore tasks}:记录每一个原子步骤的执行参数、输出及 Runner 分配详情。
\end{itemize}

\subsection{拓扑依赖存储模型 (Neo4j)}
针对 MDDG 构建,我们在 Neo4j 中设计了如下实体与关系:
\begin{itemize}
    \item \textbf{节点类型}:\texttt{Service}(微服务)、\texttt{Database}(库/表实体)、\texttt{ConfigKey}(配置项)。
    \item \textbf{关系类型}:\texttt{[:DEPEND\textunderscore ON]}(调用依赖)、\texttt{[:RELY\textunderscore BY]}(配置引用)、\texttt{[:ACCESS\textunderscore TO]}(数据读写关系)。
\end{itemize}

\subsection{监控指标与日志存储}
时序指标存储于 Prometheus 联邦中,重点关注流水线端到端耗时(Cycle Time)与变更失败率。全量执行日志通过 Fluentd 采集并索引至 Elasticsearch,支持按 \texttt{trace\textunderscore id} 进行秒级溯源。

\section{关键算法逻辑建模}

\subsection{多维度依赖图 (MDDG) 构建算法}
MDDG 算法通过聚合静态代码扫描结果与动态链路追踪数据,实时拟合系统全貌。算法定义:给定顶点集 $V=\{v_{\text{svc}}, v_{\text{cfg}}, v_{\text{db}}\}$ 与连接权重函数 $\omega(e)$,旨在构建反映系统“耦合真相”的有向加权图。

\subsection{基于权重衰减的影响分析算法 (WD-Pruning BFS)}
为降低计算复杂度,我们采用了带剪枝的广度优先搜索策略。
算法步骤:
1. 从变更源节点出发进行多层扩散;
2. 每一层传播时,风险值按 $\omega(e)$ 进行衰减;
3. 当风险分低于预设阈值 $\epsilon$ 时进行剪枝丢弃,从而在海量拓扑中快速定位核心受灾区。

\subsection{分布式分层风险评分算法}
风险分 $\text{Score} = \sum (\text{Entropy} \times \text{Weight})$。该算法引入信息熵权法,综合评估变更的文本 Diff 量、波及节点的影响度(Centrality)以及历史发布的成功率基线,输出 0-100 的客观风险评分,为自动化门禁提供决策输入。

\section{本章小结}

本章针对系统的软硬件逻辑架构、核心模块设计、数据模型及关键算法提供了详尽的概要设计说明。通过分层架构解决了复杂场景下的高伸缩性需求,通过异构数据模型兼顾了事务性与拓扑检索效率。本章所定义的架构蓝图与数学模型,不仅严格落实了前述的需求调研,更为后续章节中系统的详细实现与实验评估奠定了坚实的理论与工程基石。
