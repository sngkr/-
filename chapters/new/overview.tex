% !TeX root = ../main.tex

\chapter{系统概要设计}
\label{chap:overview}

本章给出系统的总体架构、数据流以及模块划分,为详细设计与实现提供结构化蓝图。

\section{系统总体架构}
总体架构采用分层设计:数据接入层、消息与缓存层、数据处理与存储层、查询与服务层以及运维监控层。各层职责如下:
\begin{itemize}
	\item 数据接入层:负责采集多源流量数据并做初步校验;
	\item 消息与缓存层:使用Kafka或Redis应对高并发写入与短期缓冲;
	\item 数据处理层:基于Flink/Spark实现清洗、聚合与指标计算;
	\item 存储层:冷热分离,冷存(HDFS/Object Storage)归档历史数据,热存(ClickHouse/TSDB)支撑交互式查询;
	\item 服务与展示层:提供SQL查询、API与可视化仪表盘。
\end{itemize}

\section{数据流与部署架构}
数据流从采集→消息队列→实时/离线计算→分层存储→查询服务。推荐采用容器化与Kubernetes编排以便横向扩展与运维管理,关键组件(如ClickHouse、Kafka)采用集群化部署保证可用性。

\section{模块划分与接口定义}
主要模块划分:采集适配器、接入网关、流式处理作业、批处理作业、元数据服务、查询服务与监控告警。为每个模块定义输入/输出契约、数据格式(例如统一的JSON或Parquet schema)与错误处理策略。

\section{本章小结}
概要设计提出了可扩展、可观测的系统骨架,下一章将在此基础上给出各模块的详细设计与实现要点。
