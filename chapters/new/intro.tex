% !TeX root = ../main.tex

\chapter{绪论}
\label{chap:intro}

\section{选题背景以及研究意义}
随着互联网流量、物联网设备与多源数据的快速增长,网络流量数据在运营管理、性能优化、安全检测与业务分析中扮演着越来越重要的角色。传统的关系型数据库与孤立的数据处理流程难以满足大规模流量数据的存储、查询与分析需求。构建面向流量的专用数据仓库,可以为上层的数据分析、实时监控与决策支持提供统一、可靠且高效的数据基础。

本文以流量数据仓库系统的设计与实现为研究对象,旨在解决流量数据多源异构、时序性强、数据质量参差不齐以及查询性能难以满足实时分析需求的问题。研究结果可用于提高网络运维的效率,支持业务侧的实时指标分析,并为后续基于流量的机器学习模型提供高质量的数据支撑。

\section{国内外发展现状}
国内外在数据仓库与流量分析领域已有大量研究与工程实践。传统数据仓库(如Kimball、Inmon思想)与现代大数据仓库(如Hive、ClickHouse、Snowflake等)在数据建模、存储格式与查询引擎方面提供了成熟方案。流量数据领域则出现专用的时间序列数据库(如Prometheus、InfluxDB)、列存引擎(如ClickHouse)和流处理框架(如Flink、Spark Streaming)。

尽管已有系统在分布式存储与高性能查询上取得显著进展,但针对流量数据的端到端管道(从采集、传输、清洗、建模到查询分析)仍存在挑战:采集层的协议多样性、数据清洗与脱敏的实时性、数据分层设计与历史数据管理、以及满足低延迟交互式查询的存储与索引策略等,都是当前研究与工程重点。

\section{本文研究内容和主要工作}
本文的主要工作包括:
\begin{itemize}
  \item 需求分析:梳理流量数据仓库的功能性与非功能性需求,定义关键指标与用例场景;
  \item 架构设计:提出基于分层数据仓库的总体架构,设计数据接入、存储、查询与运维模块;
  \item 详细设计与实现:实现数据接入管道(支持JDBC/ODBC、HTTP/FTP等多种传输)、ETL/ELT流程、分层数据模型与索引方案,并选取合适的存储与计算组件以保证查询性能;
  \item 测试与评估:对系统在功能性和非功能性(吞吐、延迟、并发)方面进行测试与分析,验证方案有效性;
  \item 总结与展望:总结工作贡献,分析不足并提出改进方向。
\end{itemize}

\section{论文组织架构}
全文安排如下:第1章为绪论,介绍研究背景、现状、工作与贡献;第2章介绍相关技术;第3章给出系统需求分析;第4章为系统概要设计;第5章详细设计与实现;第6章系统测试与分析;第7章总结与展望。
