% !TeX root = ../main.tex

\chapter{相关技术简介}
\label{chap:tech}

本章介绍构建流量数据仓库系统所需的关键技术,包括数据仓库的基本概念、常用框架、网络传输协议与与流量数据处理相关的实践要点。

\section{流量数据仓库相关概念}
流量数据仓库是面向网络或业务流量的专用数据存储与分析平台,通常包括数据采集层、数据存储层(冷热分层)、数据加工层(ETL/ELT)和查询分析层。与通用数据仓库相比,流量仓库更强调时序性、插入吞吐、压缩存储与高效的时间范围查询。

\section{数据仓库框架}
当前主流的数据仓库实现可分为传统数据仓库(RDBMS+ETL)、大数据仓库(分布式文件系统+列式引擎)与云数据仓库(托管服务)。流量场景下常用组件包括:
\begin{itemize}
	\item 存储:列式数据库(ClickHouse、Greenplum)、时间序列数据库(InfluxDB、TimescaleDB);
	\item 计算:批处理(Spark)、流计算(Flink)、SQL-on-Anything引擎;
	\item 元数据与调度:Hive Metastore、Airflow、Dagster等。
\end{itemize}

\subsection{数据仓库技术}
流量仓库通常采用列存与分区策略、合并小文件、向量化查询与物化视图来提升查询性能。分层存储(例如Raw、Staging、ODS、DW、MART)有助于数据治理与历史管理。

\subsection{数据仓库分层架构}
推荐的分层包括:采集层(原始流量)、清洗层(去重、时间同步、脱敏)、ODS(标准化原子记录)、DW(主题建模)与服务层(为BI/实时查询提供物化表或索引)。每层需明确定义数据契约与保留策略。

\section{网络传输协议}
流量数据来源多样,常见的传输协议与接入方式包括JDBC/ODBC、FTP/SFTP、HTTP/HTTPS、消息队列(Kafka)等,系统需支持高并发写入与断点续传、幂等处理。

\subsection{JDBC/ODBC}
适用于结构化关系型数据源的批量抽取,通常与变更数据捕获(CDC)结合以实现增量加载。

\subsection{FTP/SFTP}
传统批量数据交付方式,常用于离线数据文件的导入。需要处理文件完整性、编码与时区等问题。

\subsection{HTTP/HTTPS 与消息队列}
HTTP常用于小消息或API上报,消息队列(Kafka、Pulsar)是流量数据场景下更可靠的实时接入方案,支持高吞吐与消费分组。

\section{本章小结}
本章综述了流量数据仓库相关的概念与可选技术栈,为后续系统需求与架构设计提供技术背景与选型依据。
