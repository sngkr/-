% !TeX root = ../main.tex

\chapter{系统详细设计与实现}
\label{chap:design}

本章给出关键模块的详细设计,包括数据模型、处理流程、索引与分区策略,以及实现示例。

\section{详细设计}
\subsection{数据模型}
针对流量数据,采用宽表+时间分区的存储策略。主表包含时间戳、源/目的IP、端口、协议、字节数、会话ID和业务标签等字段。对高基数字段(如IP)使用分层维表与哈希分桶以减少热点。

\subsection{分区与索引}
使用按天或按小时的时间分区,并在热存中为常用聚合键建立稀疏索引或物化聚合视图,以加速常见的时间窗口查询。

\section{关键组件实现}
\subsection{数据接入}
实现可插拔的采集适配器:批量采集通过文件协议(FTP/SFTP),实时采集通过Kafka或HTTP推送。所有接入点统一写入消息队列并由流处理作业消费。

\subsection{流处理与批处理}
实时指标通过Flink进行流式计算,晚到数据和复杂变换通过Spark批处理完成。采用幂等输出与检查点机制保证处理的可靠性。

\subsection{存储与查询组件}
热存采用ClickHouse以获得列式存储与高速聚合,冷存采用分布式对象存储(如S3/HDFS)保存长期历史数据,ETL作业负责在两者间搬运与压缩。

\section{测试与上线流程}
在开发流程中引入单元测试、集成测试与性能回归测试。CI/CD管道包含代码检测、逐环境部署与自动化回滚策略。上线前需进行容量与压力测试,确保在预期负载下系统稳定。

\section{本章小结}
本章在概要设计基础上进一步明确了数据模型与实现细节,为下一章的测试与性能评估做好准备.
