% !TeX root = ../main.tex

\chapter{系统测试与分析}
\label{chap:testing}

本章介绍系统的测试策略、典型测试用例与性能分析结果,并针对发现的问题给出改进建议。

\section{测试策略与计划}
测试策略包括:单元测试验证核心模块逻辑,集成测试验证模块间的数据流,端到端测试覆盖典型业务场景,性能测试评估吞吐与延迟。制定测试矩阵,针对功能点与边界情况编写用例,并在不同负载水平下进行压力测试。

\section{测试用例与结果}
关键测试包括:
\begin{itemize}
	\item 写入吞吐测试:模拟高并发数据接入,评估消息队列与流处理的最大稳定吞吐;
	\item 延迟测试:测量从数据入队到可查询的端到端延迟;
	\item 查询性能:在典型聚合场景下测试ClickHouse等热存的响应时间与并发能力;
	\item 数据质量测试:注入脏数据验证清洗作业的鲁棒性与告警覆盖率。
\end{itemize}

测试结果表明:在标准集群配置下,系统可支持每秒百万级的写入事件量,并在多数常见聚合查询中实现亚秒到数秒级响应(依赖查询复杂度与时间范围)。具体数值与集群规格、分区策略密切相关,见附录性能表。

\section{问题与改进}
测试中发现的主要问题包括:热点分区导致局部性能下降、晚到数据造成统计偏差、以及部分长尾查询响应较慢。改进措施包括:更细粒度的分区策略、增加物化聚合视图、引入延迟补偿策略与查询超时机制。

\section{本章小结}
通过系统测试验证了设计目标的可达性,并针对发现的问题提出了工程改进建议以提升稳定性与性能.
