% !TeX root = ../main.tex

\ustcsetup{
  keywords  = {持续交付;多变更域;变更影响分析;渐进式发布;流水线自动化},
  keywords* = {Continuous Delivery, Multi-change Domain, Change-impact Analysis, Progressive Release, Pipeline Automation},
}

\begin{abstract}
  本文围绕“面向多变更领域的持续交付系统设计与实现”开展研究,针对跨领域、多变更场景下部署频繁、依赖复杂、回滚与验证成本高等问题,提出了一套可扩展的持续交付体系与实现方法。

  首先,分析了多变更域环境中变更的特征与传播路径,构建了基于依赖感知的变更影响分析模型;其次,设计了支持多域协同的流水线编排模型,结合分层策略实现流水线复用与隔离,降低跨域发布冲突;然后,在部署策略上引入渐进式发布与自动回滚机制,并在测试环节中融入自动化回归与契约检测以提升发布安全性;最后,基于容器化与微服务治理技术实现原型系统,并在若干典型场景下完成性能与可靠性评估。

  实验结果表明:所提方法能显著减少部署失败率与人工干预次数,缩短变更交付周期,同时在保证系统可用性的前提下提高发布频率。本文的主要贡献包括:提出变更影响分析与多域流水线编排方法、实现一套工程化持续交付原型系统并完成系统化评估。

\end{abstract}

\begin{abstract*}
  This thesis addresses the design and implementation of a continuous delivery (CD) system tailored for multi-change domains, where frequent cross-domain changes, complex dependencies, and high rollback/testing costs pose significant challenges.

  We first analyze change characteristics and propagation across domains and propose a dependency-aware change-impact analysis model. Then we design a multi-domain pipeline orchestration model that supports pipeline reuse and isolation through layered strategies, reducing cross-domain release conflicts. For deployment, we incorporate progressive release and automated rollback mechanisms, and integrate automated regression and contract checks into the testing phase to improve release safety. Finally, a prototype system is implemented using containerization and microservice governance technologies, and its performance and reliability are evaluated in representative scenarios.

  Experiments show that the proposed approach reduces deployment failure rates and manual interventions, shortens delivery cycles, and increases release frequency while maintaining system availability. The main contributions of this work are the change-impact analysis and multi-domain pipeline orchestration methods, an engineering prototype of a CD system, and a systematic evaluation.

\end{abstract*}
