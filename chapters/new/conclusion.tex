% !TeX root = ../main.tex

\chapter{总结与展望}
\label{chap:conclusion}

\section{论文工作总结}
本文针对流量数据仓库的需求,提出了基于分层架构的系统设计,并给出了关键模块的详细实现方案与工程化实践要点。实现部分包括多协议接入、流批一体的数据加工、冷热分离的存储策略以及为交互式查询优化的索引与物化视图。通过测试与评估,验证了系统在功能性与性能方面的可行性。

\section{存在的不足}
当前工作仍存在若干局限:在非常大规模(数十亿级别)历史数据的长期归档与压缩策略上尚需进一步优化;复杂跨表联接、深度历史回溯查询在成本上仍然较高;此外对隐私保护与数据合规的系统化支持需要加强。

\section{未来展望}
未来可沿以下方向推进:
\begin{itemize}
	\item 引入更多基于列式存储的压缩与分层归档策略以降低长期存储成本;
	\item 将物化视图与自动索引策略结合机器学习方法实现自适应查询加速;
	\item 加强隐私计算与脱敏流水线,支持合规的数据共享与分析;
	\item 考虑将系统拆分为更细粒度的微服务以便更灵活地弹性扩缩容和多租户支持。
\end{itemize}

本文的研究为面向流量的数据仓库工程化提供了实践参考,后续工作将聚焦于大规模场景的成本优化与智能化运维能力。
