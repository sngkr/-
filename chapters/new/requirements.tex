% !TeX root = ../main.tex

\chapter{需求分析与典型场景建模}
\label{chap:requirements}

本章首先通过对实际企业级软件交付过程的深入调研,提炼出几种典型的多变更域协同场景,深刻剖析其中蕴含的复杂依赖关系与痛点问题。在此基础上,利用形式化方法对多变更域持续交付问题进行建模,明确核心概念与逻辑关系。最后,结合场景分析与理论模型,详细阐述系统的功能性需求与非功能性需求,明确系统设计目标与约束条件。

\section{概述}

需求分析是系统设计的基石。在微服务架构与云原生技术普及的今天,软件交付早已超越了单纯的“代码发布”范畴,演变为涵盖应用代码、运行时配置、基础设施资源、数据存储模式等多个域的复杂系统工程。传统的持续交付系统多聚焦于单一应用流水线,难以有效应对跨域变更带来的协同挑战。

为了构建一套真正面向多变更域的持续交付体系,必须首先厘清多变更域场景下的核心矛盾。本章将从实际生产环境中的典型案例出发,识别不同变更域之间的耦合模式与风险传播路径,从而推导出系统所需的核心能力,如全链路依赖感知、跨域流水线编排、智能发布控制等,为后续的架构设计与系统实现提供坚实的依据。

\section{典型多变更域场景分析}

在现代分布式系统中,单一功能的上线往往牵一发而动全身。本节选取了四个最具代表性的多变更域协同场景,分析其变更流程、依赖关系及潜在风险。

\subsection{跨服务功能变更协同场景}

随着业务拆分的细化,一个用户请求往往需要经过多个微服务的协作才能完成。当业务需求发生变更时,通常需要多个服务同时修改接口或逻辑,形成了跨服务的变更依赖。

\textbf{场景描述}:
假设电商系统中新增了“预售商品”功能。该功能涉及以下服务变更:
\begin{enumerate}
    \item \textbf{商品服务(Product Service)}:需修改数据库Schema以支持“预售”标记,并提供新的API接口查询预售信息。
    \item \textbf{订单服务(Order Service)}:需并在下单流程中增加预售校验逻辑,调用商品服务的新接口。
    \item \textbf{库存服务(Inventory Service)}:需区分预售库存与现货库存,逻辑需做相应调整。
    \item \textbf{前端网关(API Gateway)}:需聚合新的预售字段并透出给前端。
\end{enumerate}

\textbf{变更依赖与挑战}:
此场景中存在严格的逻辑依赖与发布时序依赖:
\begin{itemize}
    \item \textbf{接口依赖}:订单服务依赖商品服务的新接口。若订单服务先上线,调用旧版商品服务将报错,导致下单失败。
    \item \textbf{数据依赖}:商品服务代码上线前,数据库Schema必须先完成变更,否则服务启动失败。
    \item \textbf{版本兼容性}:在灰度发布过程中,新旧版本服务可能共存。例如,新版订单服务请求打到了旧版商品服务节点,必须有降级策略或路由控制,否则会导致请求异常。
\end{itemize}

\textbf{痛点分析}:
在现有体系下,各服务往往由不同团队维护,拥有独立的流水线。协调这些服务的发布顺序全靠人工沟通(如飞书群、Excel表),极易出现“漏发”、“错序”或“未对齐灰度比例”的情况,导致线上故障。

\subsection{配置驱动的业务逻辑变更场景}

配置中心(Configuration Center)使得应用程序可以在不重新部署代码的情况下调整运行时行为。然而,配置变更并非总是无风险的,配置与代码之间存在着隐性的强耦合。

\textbf{场景描述}:
某推荐系统计划上线一种新的“协同过滤算法”。
\begin{enumerate}
    \item \textbf{代码变更}:算法团队开发了新的推荐模型代码,并已合入主干,随常规版本发布。
    \item \textbf{配置变更}:运维人员需要在配置中心新增一个开关 \texttt{enable\_collaborative\_filtering = true},并配置算法参数 \texttt{model\_path} 和 \texttt{timeout\_ms}。
\end{enumerate}

\textbf{变更依赖与挑战}:
\begin{itemize}
    \item \textbf{时序依赖}:代码必须先于配置上线。如果配置先下发,而代码中尚无处理该配置的逻辑(或者旧版本代码对新配置项解析异常),可能导致服务崩溃或行为不可预期。
    \item \textbf{值域依赖}:配置值的合法性依赖于代码实现。例如,代码中对 \texttt{timeout\_ms} 设定的范围是 0-1000ms,若配置误配为 5000ms,可能引发线程池耗尽。
    \item \textbf{全量风险}:配置下发通常是毫秒级全量生效的,一旦配置错误,所有实例同时挂掉,不仅无法通过常规的金丝雀发布规避,甚至会导致全站雪崩。
\end{itemize}

\textbf{痛点分析}:
目前的CI/CD工具常将代码流水线与配置管理隔离。代码发布有层层测试把关,而配置修改往往是运维人员在控制台“点一下”,缺乏自动化测试与依赖校验。系统无法识别“该配置项需要最低版本v1.2.0支持”,极易引发因版本不匹配导致的事故。

\subsection{基础设施升级与应用适配场景}

基础设施即代码(IaC)使得底层资源的变更也变得频繁。基础设施的升级(如Kubernetes版本升级、中间件替换)往往对上层应用产生深远影响。

\textbf{场景描述}:
平台团队计划将Kubernetes集群从v1.22升级至v1.25,同时废弃旧版Ingress API (\texttt{networking.k8s.io/v1beta1}),强制迁移到v1版本。

\textbf{变更依赖与挑战}:
\begin{itemize}
    \item \textbf{API兼容性依赖}:所有应用服务的Helm Chart或YAML文件必须修改,将Ingress定义升级为新版API。
    \item \textbf{环境依赖}:应用代码可能依赖特定版本的JDK基础镜像或Sidecar版本。基础设施升级基础镜像后,应用可能因类库冲突无法启动。
    \item \textbf{规模效应}:基础设施变更通常涉及成百上千个应用。如何确保所有应用都已完成适配?如果强制升级,未适配的应用将无法部署;如果逐个通知,沟通成本极高。
\end{itemize}

\textbf{痛点分析}:
基础设施层与应用层处于割裂状态。平台团队难以精准评估一个底层变更会影响哪些上层应用(例如,哪些应用还在使用旧版API)。通常采用“人肉排查”或“停机演练”的方式,效率低下且风险不可控。

\subsection{数据模式演进与代码同步场景}

数据是系统的核心资产。在业务快速迭代中,数据库Schema(DDL)的变更极其频繁,且回滚代价巨大。

\textbf{场景描述}:
用户中心服务需要将 \texttt{user\_id} 字段从 \texttt{int} 类型扩展为 \texttt{bigint} 以支撑用户量增长。

\textbf{变更依赖与挑战}:
\begin{itemize}
    \item \textbf{破坏性变更}:修改主键类型是破坏性变更(Breaking Change)。
    \item \textbf{代码-数据强一致性}:旧版代码只能处理 \texttt{int} 类型,新版代码才能处理 \texttt{bigint}。在发布过程中,若数据库先变更为 \texttt{bigint},旧版代码读取数据时可能发生溢出或类型转换错误。
    \item \textbf{不可逆性}:一旦数据写入了超过 \texttt{int} 上限的数值,数据库就无法简单的回滚到 \texttt{int} 类型Schema。这要求发布策略必须极其谨慎,通常采用“双写过渡”方案。
\end{itemize}

\textbf{痛点分析}:
DDL变更通常由DBA手动执行或通过独立的工单系统流转,与应用发布流水线脱节。系统缺乏对“数据库变更-应用发布”这一原子操作的编排能力,难以实现自动化的双写迁移或停机发布控制。

\section{多变更域问题建模}

为了科学地解决上述复杂场景中的协同问题,我们需要将非结构化的业务问题转化为结构化的数学模型。本节定义核心概念,并构建变更传播模型。

\subsection{核心概念定义}

\begin{definition}[变更域 (Change Domain)]
变更域是指在软件交付过程中,具有独立版本控制、独立生命周期管理且逻辑上内聚的实体集合。常见的变更域 $D$ 包括:
\begin{itemize}
    \item \textbf{应用域 ($D_{app}$)}:包含源代码、编译产物、容器镜像等。
    \item \textbf{配置域 ($D_{conf}$)}:包含环境变量、启动参数、特征开关、动态配置等。
    \item \textbf{设施域 ($D_{infra}$)}:包含Kubernetes资源(Deployment/Service)、虚拟机、存储卷、网络策略等。
    \item \textbf{数据域 ($D_{data}$)}:包含数据库Schema、索引、初始数据、消息队列Topic定义等。
\end{itemize}
\end{definition}

\begin{definition}[变更原子 (Change Atom)]
变更原子 $c$ 是指发生在一个变更域内的最小不可分割修改单元。一个变更请求 $R$ 通常由多个变更原子组成,即 $R = \{c_1, c_2, ..., c_n\}$,其中 $c_i \in D_j$。
例如,一个功能上线可能包含:修改应用代码 ($c_{app}$)、新增配置项 ($c_{conf}$) 和修改数据表结构 ($c_{data}$)。
\end{definition}

\begin{definition}[依赖关系 (Dependency)]
对于两个变更原子 $c_i$ 和 $c_j$,若 $c_i$ 的正确执行或生效前提是 $c_j$ 处于特定状态,则称 $c_i$ 依赖于 $c_j$,记作 $c_i \to c_j$。
依赖关系具有多种类型:
\begin{itemize}
    \item \textbf{时序依赖 (Sequential)}:$c_j$ 必须在 $c_i$ 之前完成执行(如 DDL $\to$ 代码部署)。
    \item \textbf{值域依赖 (Value)}:$c_i$ 的取值必须在 $c_j$ 定义的范围内(如 配置值 $\to$ 代码逻辑)。
    \item \textbf{共存依赖 (Co-existence)}:$c_i$ 和 $c_j$ 必须同时存在或同时不存在(如 Service $\to$ Deployment)。
\end{itemize}
\end{definition}

\subsection{依赖关系图模型}

基于上述定义,整个系统的状态空间可以建模为一个有向图 $G = (V, E)$。

\begin{itemize}
    \item \textbf{顶点集 $V$}:代表系统中的所有实体(服务、配置项、资源对象)。
    \item \textbf{边集 $E$}:代表实体间的依赖关系。
\end{itemize}

在持续交付过程中,我们将“变更”视为对图 $G$ 的扰动。构建 \textbf{多维依赖图 (Multi-Dimensional Dependency Graph, MDDG)} 是解决问题的关键。MDDG 不同于传统的静态调用图,它是一个异构图(Heterogeneous Graph)。

$$ MDDG = \{ G_{app}, G_{conf}, G_{infra}, E_{cross} \} $$

其中,$G_{app}$ 是应用间的调用图,$G_{conf}$ 是配置与应用的映射图,$G_{infra}$ 是设施拓扑图,$E_{cross}$ 是跨域关联边。例如,一个Pod节点(设施)运行了某个镜像(应用),挂载了某个ConfigMap(配置),连接了某个Service(设施)。

\subsection{变更风险传播模型}

基于MDDG,我们可以定义变更风险的传播路径。
设初始变更集合为 $C_{init}$,风险传播函数为 $P(v)$,表示节点 $v$ 受影响的概率或严重程度。

$$ P(v) = \alpha \cdot I(v) + \beta \cdot \sum_{u \in In(v)} w(u, v) \cdot P(u) $$

其中:
\begin{itemize}
    \item $I(v)$:节点本身的内在风险(如核心服务比边缘服务风险大)。
    \item $In(v)$:所有指向 $v$ 的前驱节点集合(即依赖 $v$ 的节点)。
    \item $w(u, v)$:依赖强度权重。强依赖(如RPC调用)权重高,弱依赖(如异步消息)权重低。
    \item $\alpha, \beta$:调节系数。
\end{itemize}

该模型揭示了风险传播的“蝴蝶效应”:底层的微小变更(如公共类库升级、基础镜像漏洞修复)可能沿着依赖链向上传导,最终导致大量顶层业务服务的不稳定。

\section{系统功能性需求}

基于典型场景分析与理论建模,本研究提出的多变更域持续交付系统需具备以下四大核心功能模块。

\subsection{多维度依赖自动分析}

系统必须具备自动化构建 MDDG 的能力,摒弃人工维护依赖关系的落后方式。

\begin{enumerate}
    \item \textbf{静态解析能力}:能够解析代码(Java/Go import)、配置(Spring @Value, YAML)、设施(K8s Manifest, Terraform)文件,提取显式依赖。
    \item \textbf{动态追踪能力}:集成APM(如SkyWalking、Jaeger)数据,捕获运行时RPC调用链,识别静态分析无法发现的动态依赖或条件依赖。
    \item \textbf{跨域关联能力}:能够将不同域的实体关联起来。例如,通过解析 Deployment YAML 将 Service 与 ConfigMap 关联;通过解析 datasource 配置将 Application 与 Database 关联。
    \item \textbf{影响域计算}:给定一个变更请求(如修改配置项A),系统能快速计算出“受影响半径”,输出受影响的服务列表及风险等级报告。
\end{enumerate}

\subsection{跨域流水线编排}

传统的单体流水线已无法满足需求,系统需支持分层、模块化的流水线编排。

\begin{enumerate}
    \item \textbf{层级化模型}:支持“变更原子流水线”(构建、部署单个组件)与“业务发布流水线”(编排多个原子流水线)。
    \item \textbf{依赖驱动的调度}:流水线引擎应具备DAG(有向无环图)调度能力,根据变更原子间的时序依赖关系(由前述模型计算得出)自动生成执行计划。例如,自动安排先执行DDL变更,再并行部署三个微服务。
    \item \textbf{原子性与事务性}:支持跨域变更的原子性。如果中间某个环节失败(如ConfigMap更新成功但Pod重启失败),能够触发自动回滚流程,将所有相关域恢复至变更前状态。
\end{enumerate}

\subsection{依赖感知的发布控制}

发布过程必须是智能的、灰度的、可观测的。

\begin{enumerate}
    \item \textbf{多级灰度策略}:支持基于流量(Header/Cookie)、地域、用户ID的精细化流量切分。灰度不仅作用于应用流量,也应尽可能作用于配置生效范围。
    \item \textbf{协同发布网关}:提供统一的发布控制平面,协调不同服务的灰度比例。例如,保证服务A的灰度版本只调用服务B的灰度版本(全链路灰度),防止环境串扰。
    \item \textbf{发布门禁(Quality Gates)}:在发布的每个阶段(Canary 1\% -> 10\% -> 100%)设置自动化检查点,检查指标(Error Rate, Latency)是否超标,依赖约束是否满足。
\end{enumerate}

\subsection{全链路变更监控与回溯}

\begin{enumerate}
    \item \textbf{统一变更事件流}:收集所有域的变更事件(Git Commit, K8s Event, Config Change, SQL Execution)至统一的时间序列数据库。
    \item \textbf{变更与故障关联}:当系统出现故障时,能迅速在时间轴上定位最近发生的变更事件,并结合依赖图推荐可能的“嫌疑变更”。
    \item \textbf{配置漂移检测}:定期扫描实际运行环境与Git仓库的差异,发现并告警“有人在生产环境直接修改了配置但未提交代码”的违规行为。
\end{enumerate}

\section{系统非功能性需求}

除了功能完善,作为企业级核心基础设施,系统还必须满足严格的非功能性指标。

\subsection{可扩展性 (Scalability)}
\begin{itemize}
    \item \textbf{接入能力}:系统应能支撑数千个微服务、数万个配置项的依赖分析,依赖图构建算法的时间复杂度应控制在合理范围(如 $O(N \log N)$)。
    \item \textbf{插件化架构}:适应异构技术栈。不同的编程语言(Java/Go/Node)、不同的基础设施(K8s/VM/Serverless)应能通过插件方式轻松接入,无需修改核心代码。
\end{itemize}

\subsection{高可用性 (Availability)}
持续交付系统是研发效能的“生命线”。如果发布系统宕机,将导致无法修复线上紧急Bug。
\begin{itemize}
    \item \textbf{系统自身高可用}:核心组件需集群部署,支持多活容灾。
    \item \textbf{故障降级}:在依赖分析服务不可用时,应允许降级为人工审批模式继续发布,而不是完全阻断发布流程。
\end{itemize}

\subsection{安全性 (Security)}
\begin{itemize}
    \item \textbf{权限隔离}:支持基于RBAC(Role-Based Access Control)的细粒度权限控制。开发人员只能变更自己服务的代码,不能随意修改生产环境的基础设施或敏感配置。
    \item \textbf{审计合规}:所有变更操作必须留痕,满足金融级审计要求。记录“谁、在什么时间、修改了哪个域的什么内容、审批人是谁”。
    \item \textbf{秘钥管理}:敏感配置(如DB密码)必须加密存储与传输,严禁明文出现在流水线日志中。
\end{itemize}

\subsection{易用性 (Usability)}
\begin{itemize}
    \item \textbf{开发者体验}:提供可视化的依赖视图与发布大盘,让复杂的依赖关系“看得见、摸得着”。
    \item \textbf{声明式配置}:尽可能采用声明式(Declarative)而非指令式(Imperative)的交互方式,降低用户的学习成本与操作负担。
\end{itemize}

\section{本章小结}

本章深入探讨了多变更域持续交付面临的现实挑战。通过对跨服务、配置、基础设施、数据等典型协同场景的剖析,我们揭示了现有单体交付体系在应对复杂依赖时的局限性。随后,我们建立了包含变更域、依赖关系图、风险传播在内的理论模型,为问题的求解提供了数学工具。

在此基础上,我们明确了系统的功能性需求,重点在于“多维依赖分析”、“跨域编排”、“智能发布”与“全链路监控”四大支柱;同时界定了性能、安全等非功能性约束。这些分析结论为下一章的系统详细架构设计指明了方向,奠定了坚实的理论基础和工程目标。

% End of requirements.tex
