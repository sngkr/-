% !TeX root = ../main.tex

\chapter{面向多变更领域的持续交付系统需求分析}
\label{chap:requirements}

系统需求分析是开发“面向多变更领域的持续交付系统”的基础。本章旨在通过分析现代软件开发中微服务架构、容器化设施及云原生应用的复杂交付过程,准确确立系统的设计目标与功能边界。在多变更领域背景下,交付过程涉及代码变更、配置调整以及数据库模式演进等多个异构维度。本系统的核心价值在于:为任何可能影响服务可用性的异构变更提供统一的自动化评估与交付路径,而非强制建立复杂的跨域依赖。本章分析内容涵盖了如何实时感知多维变更信号、如何结合特定应用上下文执行自动化合规预审与可用性验证,以及如何构建高效的结果反馈闭环。本章将从研发、测试人员及系统管理员的角色视角出发,详细解构其在变更交付与平台治理中的功能性需求,并从高性能处理、全方位安全性与分布式可靠性三个维度分析非功能性需求。通过建立严密的需求规格体系,为后续的系统架构设计及核心算法实现提供定量与定性的指导指标。

\section{功能性需求分析}


\subsection{用户需求分析}

面向多变更领域的持续交付系统主要解决代码、配置、数据库等异构变更在统一时空下的自动化评估与交付问题。在复杂的微服务环境下,影响服务运行态的因素不仅限于代码,配置调整或数据库模式的演进同样是引发线上事故的高风险点。本系统的核心价值在于:不论变更源自哪个领域,系统都能验证变更对当前或预设服务可用性的影响。

普通用户(研发、测试与应用运维人员)是交付任务的发起者。在多变更环境下,用户的核心痛点在于:当非代码类变动(如数据库变更或配置调整)独立发生时,难以快速、准确地验证这些变更是否与特定的应用版本兼容。例如,当数据库 DDL 变更发生时,用户并不希望仅停留在 SQL 规约的手动审核,而是需要系统能够自动关联指定的应用代码版本,触发相应的流水线进行合规扫描与可用性评估,从而验证变更是否会导致服务启动异常或逻辑错误。

因此,普通用户的核心需求在于实现“变更驱动的可用性评估”。系统应能根据变更类型灵活触发流水线:对于纯代码变更,执行标准的构建发布流;对于数据库或配置变更,则需自动结合指定的应用基线版本进行兼容性验证。这种机制旨在确保任何影响服务的变更都能在上线前通过全自动的流水线得到充分验证,从而降低由于人工评估缺失而引入的线上风险。

系统管理员的需求则侧重于平台的治理、策略一致性与资源调度。管理员需要为不同维度的变更定义差异化的检查规则与审批流。例如,数据库变更可能需要更严格的 DBA 预审插件,而配置变更则侧重于格式规范与环境隔离的检查。由于多种变更可能会频繁触发不同规模的评估任务,管理员需要系统具备高效的资源感知与调度分析能力,保障流水线在高峰期的稳定响应。

系统的用例交互关系如图 \ref{fig:usecase} 所示。该图详细展示了普通用户与系统管理员在交付全生命周期中的职能路径划分:普通用户关注于异构变更任务的可用性自动化验证,而系统管理员则侧重于全局性质的合规治理、规则引擎维护与资源配额利用率的最优匹配。

\begin{figure}[!ht]
    \centering
    \begin{plantuml}
@startuml
left to right direction
skinparam packageStyle rectangle
skinparam shadowing false
skinparam defaultFontName "SimSun"

actor "普通用户(研/测/运维)" as User
actor "系统管理员" as Admin

rectangle "面向多变更领域的持续交付系统" as CDSystem {

  (提交代码变更)              as UC_Code
  (提交配置/参数变更)          as UC_Config
  (提交数据库结构变更)         as UC_DB
  (选择目标应用版本)           as UC_Version
  (触发变更评估流水线)         as UC_Trigger
  (自动化合规检查)             as UC_Compliance
  (可用性评估)                 as UC_Availability
  (查看评估报告与结果)         as UC_Report

  (配置检查规则集)             as UC_Rule
  (配置审批流程策略)           as UC_Approval
  (管理评估任务资源配额)       as UC_Quota

  ' 普通用户用例
  User -- UC_Code
  User -- UC_Config
  User -- UC_DB
  User -- UC_Version
  User -- UC_Trigger
  User -- UC_Report

  ' 触发–检查–评估 的用例关系
  UC_Trigger ..> UC_Compliance      : <<include>>
  UC_Compliance ..> UC_Availability : <<include>>

  ' 管理员配置类用例
  Admin -- UC_Rule
  Admin -- UC_Approval
  Admin -- UC_Quota

  ' 检查/评估对管理员配置的依赖
  UC_Compliance ..> UC_Rule         : <<include>>
  UC_Trigger ..> UC_Approval        : <<include>>
  UC_Trigger ..> UC_Quota           : <<include>>
}

@enduml
    \end{plantuml}
    \caption{系统用例图}
    \label{fig:usecase}
\end{figure}


\subsection{系统整体功能分析}

面向多变更领域的持续交付系统的核心价值在于通过各功能模块间的深度闭环协作,解决多域变更在分布式环境下的协同一致性问题。系统的运行逻辑并非简单的线性流程,而是一个集实时感知、智能决策、并行执行与主动反馈于一体的闭环生命周期,如图 \ref{fig:cd_flow} 所示。

\begin{figure}[htbp]
    \centering
    \begin{plantuml}
@startuml
skinparam componentStyle rectangle
skinparam shadowing false
skinparam defaultFontName "SimSun"

skinparam node {
  BackgroundColor white
  BorderColor #666666
}

skinparam component {
  BackgroundColor #FFE5CC
  BorderColor #E67E22
}

' 定义模块
component "触发感知与处理模块" as Trigger
component "检查能力模块" as Check
component "引擎模块" as Engine
component "反馈模块" as Feedback

' 触发子项 (左侧)
rectangle "代码变更" as Code
rectangle "配置变更" as Config
rectangle "数据变更" as Data
Code -right- Trigger
Config -right- Trigger
Data -right- Trigger

' 检查子项 (上方)
rectangle "配置中心" as ConfCenter
rectangle "能力拓展中心" as CapCenter
rectangle "Pipeline中心" as PipeCenter
ConfCenter -down- Check
CapCenter -down- Check
PipeCenter -down- Check

' 引擎子项 (右侧)
rectangle "Pipeline引擎" as PipeEngine
PipeEngine -left- Engine

' 反馈子项 (下方)
rectangle "实时反馈" as RealTime
rectangle "OpenApi/MQ 模块" as MQ
rectangle "ResultCenter" as ResultCenter
RealTime -up- Feedback
MQ -up- Feedback
ResultCenter -up- Feedback

' 循环关系
Trigger -[hidden]up- Check
Check -[hidden]right- Engine
Engine -[hidden]down- Feedback
Feedback -[hidden]left- Trigger

Trigger .up.> Check
Check .right.> Engine
Engine .down.> Feedback
Feedback .left.> Trigger

@enduml
    \end{plantuml}
    \caption{系统核心功能模块交互关系图}
    \label{fig:cd_flow}
\end{figure}

系统的每一个交付任务流均体现了从外部异构变更输入到基础设施自动化操作输出的完整映射。基于多变更领域的业务特征,系统核心的交付用例交互流程在宏观上可解构为以下三个关键技术环节:

(1) 变更感知与执行环境热启动:当研发或运维主体通过 Git 代码库、配置中心或数据库发起变更请求时,系统的“多维触发与语义解析”模块利用多协议监听机制(如 Webhook、MQ 或轮询)捕获信号。系统需通过深度解构提取变更任务的业务语义(如分支标识、提交 ID、变更域类型),并以此触发布署。随后,资源管理中心根据计算池实时水位动态纳管执行容器,并与凭据库建立加密连接,确保流水线在运行前已完成规约加载与身份准入。

(2) 逻辑编排驱动与任务并行运行:进入运行态后,系统通过“决策面”与“控制面”的分层协作完成交付。检查能力模块作为全系统的逻辑“大脑”,负责依据变更类型(代码/配置/数据)动态匹配合规包,并驱动流水线节点的流转与逻辑编排,执行“同步拦截”预审。与此同时,流水线执行中心作为物理执行单元,负责处理具体的脚本解析、环境指令分发及操作轨迹记录。这种架构设计实现了逻辑编排与物理执行的解耦,确保了异构变更下交付逻辑的灵活性与确定性。

(3) 主动反馈通知与状态归档闭环:在评估任务执行完毕或触发异常后,系统立即启动状态回收与反馈机制。反馈模块主动聚合执行日志、风险门禁报告及执行结论,通过 IM 工具等渠道将其即时推送到相关责任主体。随后,执行引擎触发清理指令,释放占用的 K8s 节点、临时存储及计算容器,维持资源池的动态配额。最后,系统将本次变更的全量元数据(包含执行审计、时序快照及运行产物)进行持久化归档,为后续的合规溯源、效能度量与性能分析提供核心数据资产。

综上所述,为了实现上述多维变更的交付闭环,系统的功能规格体系主要由以下七个互联的逻辑模块共同支撑:负责安全准入的“基础鉴权与中心门户”、负责信号捕获的“多维触发与语义解析”、充当逻辑大脑的“检查能力适配管理”、负责物理执行的“流水线执行中心”、构建主动观测的“全域监控与主动反馈”、负责资产映射的“资源池与敏感凭据管理”、以及支持业务扩展的“标准开放平台接口”。

\subsection{注册登录功能需求分析}

注册、登录及账号管理是系统的基础功能,主要负责确定用户身份并赋予相应的权限。本节主要分析用户注册、登录验证以及密码管理的业务逻辑,其功能逻辑如图 \ref{fig:login_usecase} 所示。

\begin{figure}[htbp]
    \centering
    \begin{plantuml}
@startuml
left to right direction
skinparam packageStyle rectangle
skinparam shadowing false
skinparam defaultFontName "SimSun"

actor "研发/运维人员" as User

rectangle 注册登录功能模块 {
  (用户注册与资源初始化) as Register
  (深度身份认证) as Login
  (凭据主动更新与吊销) as Pwd

  User -- Register
  User -- Login
  User -- Pwd

  Register ..> (资源空间预配) : <<include>>
  Register ..> (邮箱激活验证) : <<include>>

  Login <.. (多因素鉴权 MFA) : <<extend>>
  Login ..> (令牌续约管理) : <<include>>

  Pwd ..> (全像令牌联级吊销) : <<include>>
}
@enduml
    \end{plantuml}
    \caption{注册登录功能用例图}
    \label{fig:login_usecase}
\end{figure}

用户注册功能用于新成员加入系统并获取初始权限。在注册过程中,系统会校验用户填写的用户名、密码及企业邮箱。注册成功后,系统会根据用户所在的部门,自动分配基本的访问权限,例如查看所属项目的代码仓库或使用基础流水线,方便用户快速开始工作。

\begin{table}[htbp]
\centering
\caption{用户注册用例表}
\label{tab:reg_spec}
\begin{tabular}{|l|p{10cm}|}
\hline
\textbf{用例名称} & 用户注册 \\ \hline
\textbf{用例编号} & UC-01-REG \\ \hline
\textbf{参与者} & 研发/运维人员 \\ \hline
\textbf{前置条件} & 系统页面正常打开 \\ \hline
\textbf{基本流} & 1. 用户填写姓名、密码、邮箱及部门;2. 系统核对信息唯一性与格式;3. 系统分配角色并保存记录;4. 提示注册成功。 \\ \hline
\textbf{备选流} & 2a. 用户名已重复:提示重新填写。 \\ \hline
\textbf{后置条件} & 用户账号创建成功。 \\ \hline
\end{tabular}
\end{table}

登录功能用于核实用户身份并维持会话。系统采用令牌机制管理登录状态,用户无需反复输入密码。登录时系统会验证账号密码,并支持开启多因素认证(MFA)来提高安全性。登录成功后,用户获得访问令牌,后续操作都将基于该令牌进行权限判别。

\begin{table}[htbp]
\centering
\caption{用户登录用例表}
\label{tab:login_spec}
\begin{tabular}{|l|p{10cm}|}
\hline
\textbf{用例名称} & 用户登录 \\ \hline
\textbf{用例编号} & UC-02-AUTH \\ \hline
\textbf{参与者} & 已注册用户 \\ \hline
\textbf{前置条件} & 账号处于正常状态 \\ \hline
\textbf{基本流} & 1. 用户输入账号密码;2. 系统验证数据正确性;3. 发放登录令牌;4. 进入系统主页。 \\ \hline
\textbf{备选流} & 2b. 密码输入错误:允许在次数限制内重试。 \\ \hline
\textbf{后置条件} & 用户成功登录系统。 \\ \hline
\end{tabular}
\end{table}

修改密码功能可以保护账号安全。当用户更改密码后,系统会更新加密后的密码,并让之前发放的所有登录令牌全部失效。这样可以确保修改密码后,旧的会话被强制退出,用户必须使用新密码重新登录。

\begin{table}[htbp]
\centering
\caption{修改密码用例表}
\label{tab:pwd_spec}
\begin{tabular}{|l|p{10cm}|}
\hline
\textbf{用例名称} & 修改密码 \\ \hline
\textbf{用例编号} & UC-03-REVOKE \\ \hline
\textbf{参与者} & 已登录用户 \\ \hline
\textbf{前置条件} & 用户成功登录并进入系统 \\ \hline
\textbf{基本流} & 1. 用户选择修改密码;2. 系统验证旧密码正确性;3. 保存新密码;4. 让所有旧令牌失效。 \\ \hline
\textbf{后置条件} & 旧密码失效,用户需重新登录。 \\ \hline
\end{tabular}
\end{table}

\subsection{系统管理功能需求分析}

系统管理员主要负责整个交付平台的稳定运行、安全合规以及资源调度。在多部门、多用户并发使用的情况下,管理员需要对原本零散的交付过程进行集中管控,主要工作包括配置权限规则、动态管理计算资源以及记录所有的操作流水,如图 \ref{fig:admin_usecase} 所示。

\begin{figure}[htbp]
    \centering
    \begin{plantuml}
@startuml
left to right direction
skinparam packageStyle rectangle
skinparam shadowing false
skinparam defaultFontName "SimSun"

actor "系统管理员" as Admin

rectangle 治理与资源管理平面 {
  package "权限管理" {
    Admin -- (用户权限与策略配置)
    Admin -- (多方审批流程定义)
  }
  
  package "资源管理" {
    Admin -- (K8s集群与节点纳管)
    Admin -- (执行环境配额管理)
    Admin -- (凭据与密钥加密维护)
  }
  
  package "操作审计" {
    Admin -- (流水线执行日志追溯)
    Admin -- (变更影响评估与监控)
  }
  
  (用户权限与策略配置) ..> (凭据与密钥加密维护) : <<secure>>
}
@enduml
    \end{plantuml}
    \caption{系统管理员日常管理用例图}
    \label{fig:admin_usecase}
\end{figure}

管理员的具体管理工作可以分为以下三个方面:

(1) 权限配置与任务隔离:由于交付过程涉及研发、DBA 和运维等多个角色,管理员需要设置精细的权限逻辑,确保不同项目组、不同环境(如开发、测试、生产)之间的操作边界清晰。例如,对于敏感的生产环境变更,可以设置“多人在线确认”规则,只有各方审批通过后,流水线才会正式开始执行,防止出现人为误操作。

(2) 资源调度与水位监控:管理员需要统一管理系统后端的 K8s 集群等计算资源。由于发布任务往往具有突发性,系统需要能够实时感知当前资源的忙闲程度,并根据业务的重要性自动分配资源。在任务高峰期,管理员可以通过调整配额来保证核心业务不排队,提高平台的整体运行效率。

(3) 全程记录与故障排查:为了满足安全审计的要求,系统会记录每一次变更的完整过程,包括谁在什么时候发起了何种变更,以及变更产生的具体影响。当线上出现故障时,管理员可以利用系统提供的时间轴视图,快速定位是哪次变更引发的问题,从而缩短故障处理时间,维持系统稳定。

\subsection{多触发源感知功能需求分析}
系统具备一种全方位的变更感知机制,不仅能处理传统的代码修改,还扩展到了包括数据库表结构的变化、配置文件的修订、以及组件依赖(如 GPT 组件的 Prompt 语义)的变化。系统通过实时监测这些触发因素,确保交付流程能与项目需求保持同步,保障产品的质量。

系统能够感知的触发源主要包括:

(1) 代码变更:系统需实时监控代码库的变动,包括新增、修改、删除等操作。

(2) 数据库变更:数据库表结构的变更、索引调整等操作都可能影响软件功能,系统需捕捉这些变化并评估对服务的影响。

(3) 配置文件变更:如环境配置、应用配置的变化,也是触发交付流程的重要因素。

(4) 依赖组件变更:如 GPT 组件的 Prompt 语义变更,这类变更涉及到核心业务逻辑,系统需能识别并进行验证。

用户提交变更后,系统通过感知模块捕获事件并触发流水线。多触发源感知用例图如图 \ref{fig:trigger_usecase} 所示。



\begin{figure}[ht]
    \centering
    \begin{plantuml}
@startuml
left to right direction
skinparam shadowing false
skinparam defaultFontName "SimSun"

actor "研发/运维人员" as User
actor "外部系统" as Ext

rectangle 多触发源感知功能 {
  User -- (代码提交感知)
  User -- (数据库规范变更感知)
  User -- (配置文件修订感知)
  User -- (GPT Prompt语义变更感知)
  Ext -- (Webhook/MQ信号捕获)
  
  (代码提交感知) ..> (任务信息解析) : <<include>>
  (GPT Prompt语义变更感知) ..> (任务信息解析) : <<include>>
  (Webhook/MQ信号捕获) ..> (流量削峰填谷) : <<include>>
}
@enduml
    \end{plantuml}
    \caption{多触发源感知功能用例图}
    \label{fig:trigger_usecase}
\end{figure}

\subsection{智能化与插件化流水线功能需求分析}
系统需要提供插件化的检查能力,允许开发团队参照标准化的接口创建自定义的检查脚本,以满足不同业务的需求。系统仅对插件的输入输出进行了规范,使得内容的定制变得平实和便捷。

为了更有效地管理这些检查任务,系统支持为每个检查环节添加标签属性。对于标准的检查(如分支规范、单元测试等),系统可以自动感知并排列执行顺序。而对于自定义的检查,系统通过分析标签来实现分类管理,保证检查顺序的正确性,其功能逻辑如图 \ref{fig:check_usecase} 所示。

\begin{figure}[ht]
    \centering
    \begin{plantuml}
@startuml
left to right direction
skinparam shadowing false
skinparam defaultFontName "SimSun"

actor "开发团队" as Dev
actor "管理员" as Admin

rectangle 智能化与插件化流水线 {
  Dev -- (开发自定义检查插件)
  Dev -- (配置任务标签属性)
  Admin -- (定义检查接口规范)
  Admin -- (维护标准插件库)
  
  (配置任务标签属性) ..> (执行顺序自动编排) : <<include>>
  (管理检查插件库) .> (插件动态平滑加载) : <<include>>
}
@enduml
    \end{plantuml}
    \caption{智能化与插件化流水线功能用例图}
    \label{fig:check_usecase}
\end{figure}

智能化流水线设计不仅要考虑检查的全面性,还要兼顾执行效率。系统在推荐流水线配置时,会综合考虑各环节的标签,确保各个阶段能够正确地串行或并行执行。

\subsection{引擎模块功能需求分析}
引擎模块是全系统的物理执行中心,主要负责将编排好的逻辑转化为具体的脚本指令并付诸执行。引擎需要具备解析交付脚本的能力,并将这些脚本分发到对应的远程执行机上。

在执行过程中,引擎需要实时监控任务的进度和状态。为了提高系统的效率,引擎还会分析各检查环节的历史执行耗时,优先安排耗时较短的任务先运行,从而缩短整体的反馈时间。此外,引擎还需支持任务的重试和环境的清理,确保每一次执行都是在干净、可靠的环境中进行的,其工作逻辑如图 \ref{fig:engine_usecase} 所示。

\begin{figure}[ht]
    \centering
    \begin{plantuml}
@startuml
left to right direction
skinparam shadowing false
skinparam defaultFontName "SimSun"

actor "脚本解析器" as Parser
actor "执行机集群" as Runner

rectangle 引擎模块功能 {
  Parser -- (下发原子执行指令)
  Parser -- (申请计算节点资源)
  Runner -- (执行脚本并回传状态)
  
  (下发原子执行指令) ..> (任务重试/自愈) : <<extend>>
  (执行脚本并回传状态) ..> (环境自动清理) : <<include>>
}
@enduml
    \end{plantuml}
    \caption{引擎模块执行功能用例图}
    \label{fig:engine_usecase}
\end{figure}

\subsection{反馈与观测功能需求分析}
反馈模块负责将流水线的执行日志、合规报告及结果信息进行汇总,并主动推送给相关人员。系统需要支持多种通知渠道(如 IM 工具),并将所有的历史数据持久化存储。这些数据方便用户后续进行性能趋势分析和审计对账,确保交付过程透明可见,其功能逻辑如图 \ref{fig:feedback_usecase} 所示。

\begin{figure}[ht]
    \centering
    \begin{plantuml}
@startuml
left to right direction
skinparam shadowing false
skinparam defaultFontName "SimSun"

actor "研发/测试人员" as User
actor "企业协作平台" as Platform

rectangle 反馈与观测功能 {
  User -- (实时查询执行日志)
  User -- (查看合规分析报告)
  Platform -- (多渠道结果推送)
  
  (查看合规分析报告) ..> (变更历史数据持久化) : <<include>>
  (实时查询执行日志) ..> (异常预警订阅) : <<extend>>
}
@enduml
    \end{plantuml}
    \caption{反馈与观测功能用例图}
    \label{fig:feedback_usecase}
\end{figure}

\subsection{开放平台功能需求分析}
系统旨在构建一个开放的工具集,暴露一系列外部分接口,赋予研发人员更高的灵活性来利用系统资源。接口的设计充分考虑了易用性和可扩展性,方便用户在开发生命周期的不同阶段进行交互。

在接口安全方面,系统实现了精细的鉴权机制。由于部分接口是对外开放的,为了保证安全性,只有经过授权的用户才能发送请求,从而有效防止数据泄露风险。此外,系统还实施了限流策略,在高峰时段会限制请求次数,防止系统过载,确保为用户提供稳定的服务。\begin{figure}[ht]
    \centering
    \begin{plantuml}
@startuml
left to right direction
skinparam shadowing false
skinparam defaultFontName "SimSun"

actor "第三方系统" as External
actor "系统管理员" as Admin

rectangle 开放平台功能 {
  External -- (调用 OpenAPI 获取数据)
  Admin -- (配置接口限流策略)
  Admin -- (管理K8s/凭据资源)
  
  (调用 OpenAPI 获取数据) ..> (精细化身份鉴权) : <<include>>
  (调用 OpenAPI 获取数据) ..> (分级访问隔离) : <<include>>
}
@enduml
    \end{plantuml}
    \caption{开放平台功能用例图}
    \label{fig:manage_usecase}
\end{figure}

\section{非功能性需求分析}

除了上述的具体功能外,系统还需要在性能、安全和可靠性方面达到一定的标准,以保证在企业大规模使用场景下的稳定运行。

\subsection{高性能处理需求}

在实际使用中,系统需要同时处理大量来自不同部门的发布任务。尤其是在大规模上线前,系统压力会骤增,因此高性能需求主要体现在以下方面:

(1) 任务响应效率:当用户提交变更后,系统应能快速启动执行环境,减少排队等待时间。通过提前准备执行镜像等手段,确保任务能在秒级开始运行,提高研发反馈效率。

(2) 高并发支撑能力:系统需要支持数千个任务同时运行。当资源紧张时,系统应能自动识别任务优先级,优先保证核心业务的发布任务,而将非紧急任务延后处理,确保核心链路畅通。

\subsection{安全性与访问控制需求}

安全性对于持续交付系统至关重要,必须保护账号安全并确保权限受控。具体要求如下:

(1) 关键操作二次验证:对于涉及生产环境等敏感环节的操作,系统应强制要求用户进行二次身份确认(如输入手机验证码),防止账号密码泄露导致的安全风险。

(2) 动态权限管理:系统应确保用户仅在需要时获取权限。平时大多数用户只拥有基础查看权限,只有在执行已审批的特定任务时,系统才临时赋予其对应的操作凭据,任务结束即收回,最大限度降低安全隐患。

\subsection{可靠性与故障恢复需求}

由于系统依赖多种外部服务,必须具备较强的抗风险能力:

(1) 故障容忍能力:当下游服务或网络出现暂时不稳定时,系统不应直接中断任务。系统应具备自动重试机制,并在后台记录当前状态,待环境恢复后自动继续执行。

(2) 任务断点恢复能力:对于运行时间较长的发布任务,如果中途因服务器重启等意外情况中断,系统需能准确记录执行进度。当系统恢复后,任务应能从中断点继续运行,避免重复执行或产生数据不一致的问题。

\section{本章小结}

本章对面向多变更领域的持续交付系统进行了全面的需求分析。通过对研发、运维及管理员等核心角色的需求调研,明确了系统在触发感知、流水线编排、引擎执行、反馈观测以及开放平台等五个方面的功能需求。同时,针对系统在高性能、安全隔离与故障自愈等方面的非功能性要求进行了详细说明。本章建立的需求规格体系,为后续的系统总体架构设计与核心功能实现提供了重要依据。
