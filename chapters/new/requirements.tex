% !TeX root = ../main.tex

\chapter{流量数据仓库系统需求分析}
\label{chap:requirements}

本章围绕系统的功能性与非功能性需求进行分析,明确设计时需满足的业务场景与技术约束。

\section{功能性需求}
主要功能包括:
\begin{itemize}
	\item 数据接入:支持多种协议(JDBC/ODBC、FTP、HTTP、Kafka),能够处理批量与流式数据;
	\item 数据清洗与标准化:完成字段映射、时间对齐、去重与脱敏;
	\item 数据存储与分层管理:实现Raw/ODS/DW/MART分层存储与生命周期策略;
	\item 数据建模与指标计算:提供维度建模、指标定义与物化视图能力;
	\item 查询与分析:支持交互式SQL、聚合查询与时序分析接口;
	\item 运维与监控:提供数据质量监控、作业监控、告警与审计功能。
\end{itemize}

\section{非功能性需求}
关键非功能需求包括:
\begin{itemize}
	\item 性能:满足高并发写入与低延迟查询(例如秒级聚合响应或准实时指标);
	\item 可用性:支持高可用部署与故障自动切换;
	\item 可扩展性:计算与存储按需扩展,支持水平扩展;
	\item 安全性:数据传输加密、访问控制与脱敏策略;
	\item 可维护性:规范的数据血缘、元数据管理与自动化运维能力。
\end{itemize}

\section{约束与假设}
系统假设已有稳定的数据源与网络基础设施,且允许部署分布式存储与计算组件。若在受限环境(如仅单机部署)需降低设计复杂度并调整性能预期。

\section{本章小结}
需求分析为后续架构与实现提供了明确目标,设计阶段需围绕性能、可靠性与数据治理三个重点展开权衡.
