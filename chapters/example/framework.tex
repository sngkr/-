% !TeX root = ../../main.tex

\chapter{绪论}
\label{chap:intro}

\section{选题背景与研究意义}
\subsection{持续交付面临的新挑战:从单体应用到多变更域}
\subsection{多变更域的定义与特征}
\subsection{研究意义}

\section{国内外研究现状}
\subsection{持续交付流水线与编排技术研究}
\subsection{跨域变更影响分析与依赖建模研究}
\subsection{复杂系统发布策略与回滚机制研究}
\subsection{现有研究在多变更域场景下的不足}

\section{研究目标与主要内容}
\subsection{研究目标:构建面向多变更域的高效、安全交付体系}
\subsection{主要研究内容}

\section{论文主要创新点}
\subsection{多维度统一依赖建模方法}
\subsection{基于图神经网络的跨域变更影响分析}
\subsection{依赖感知的多变更协同发布调度算法}

\section{论文组织结构}

\chapter{相关技术与理论基础}
\label{chap:related_work}

\section{云原生与微服务架构技术}
\subsection{容器编排 (Kubernetes) 与声明式资源管理}
\subsection{服务网格 (Service Mesh) 在多域治理中的应用}

\section{持续交付关键技术}
\subsection{流水线即代码 (Pipeline as Code)}
\subsection{GitOps:多变更的版本化管理}

\section{复杂依赖分析与决策算法}
\subsection{图神经网络 (GNN) 与依赖图谱}
\subsection{约束满足问题 (CSP) 在多任务协同调度中的应用}

\section{本章小结}

\chapter{面向多变更域的需求分析与场景建模}
\label{chap:requirements}

\section{多变更域环境下的问题定义}
\subsection{多变更域的构成}
\subsection{跨域变更的典型特征}

\section{典型多变更场景建模}
\subsection{场景一:业务代码与数据库Schema的协同变更}
\subsection{场景二:微服务配置与基础设施资源的联动更新}
\subsection{场景三:跨服务接口变更引发的级联影响}

\section{系统需求分析}
\subsection{核心功能需求}
\subsection{系统非功能性需求}

\section{本章小结}

\chapter{面向多变更域的系统架构与关键算法}
\label{chap:system_design}

\section{系统总体架构设计}
\subsection{系统逻辑架构:以多变更管理为核心的四层架构}
\subsection{基本工作流程}

\section{多维度跨域依赖建模方法}
\subsection{依赖关系的分类与抽象}
\subsection{异构依赖图谱的构建与动态维护机制}

\section{跨域变更影响分析算法}
\subsection{基于GNN的依赖特征学习}
\subsection{多源变更传播路径识别与风险量化评估}

\section{多变更协同流水线编排与发布调度}
\subsection{分层化流水线编排模型}
\subsection{基于CSP的依赖感知发布调度策略}

\section{本章小结}

\chapter{持续交付原型系统设计与实现}
\label{chap:implementation}

\section{系统技术架构与选型}
\subsection{核心组件设计:基于K8s Operator的变更控制器}
\subsection{数据存储与处理方案}

\section{依赖分析服务模块实现}
\subsection{多源依赖采集器}
\subsection{影响分析引擎实现}

\section{多变更流水线编排模块实现}
\subsection{自定义流水线CRD设计}
\subsection{跨域协同编排控制器实现}

\section{协同发布控制与回滚模块实现}
\subsection{基于网关的全链路灰度发布实现}
\subsection{依赖感知的联动回滚机制实现}

\section{本章小结}

\chapter{实验评估与案例分析}
\label{chap:evaluation}

\section{实验环境与设计}
\subsection{实验平台搭建与多变更负载生成}
\subsection{评估指标体系}

\section{算法性能评估}
\subsection{变更影响分析准确性验证}
\subsection{协同调度算法与常见算法对比}

\section{典型多变更场景案例验证}
\subsection{案例与分析:代码-配置-数据库多维变更的协同发布}
\subsection{案例与分析:基础设施变更引发的跨域故障回滚}

\section{系统综合效能分析}

\section{本章小结}

\chapter{总结与展望}
\label{chap:conclusion}

\section{论文主要工作与贡献}
\section{研究局限性}
\section{未来工作展望}
